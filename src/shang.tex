\begin{frame}
  \frametitle{Equations}
  \framesubtitle{Based on the amphoteric Donnan (\it{amph-D}) model}
  \begin{columns}
    \begin{column}{0.5\textwidth}
      % Concentration
      \[
      \frac{\partial \varepsilon_{ma} c}{\partial t}
      + \nabla \cdot \left( - D_{eff} \nabla{c}  \right)
      = \dot{S}
      \]

      % Source Term
      \[
      \dot{S}(\phi_s - \phi_e, c) = - \frac{\varepsilon_{mi}}{F}
      \frac{\partial c_{ions,\,mi}}{\partial t}
      \]

      % Current conservation
      \[
      \nabla \cdot (- \kappa_{eff} \nabla \phi_e) = \varepsilon_{mi} \dot{Q} + i_L
      \]

      % Salt concentration in flow channel
      \[
      \varepsilon_{FC} + \nabla{(\vec{v_s}) c}
      + \nabla{(- D_{eff,FC} \nabla{c})}
      \]
    \end{column}

    \begin{column}{0.5\textwidth}  %%<--- here

      % Charge Balance
      \[
      \rho_e - \rho_{mi} - \rho_{chem} = 0
      \]

      % Potential Balnace
      \[
      \phi_s - \phi_e = \Delta \phi_S + \Delta \phi_D
      \]

      \[
      \Delta \phi_S = \frac{F \rho_e}{\rho_{electrode} C_S}, \,
      \Delta \phi_D = \frac{\rho_mi}{2c}
      \]

      \[
      (c_{ions,\,mi})^2 = (\rho_{mi})^2 + (2 c F)^2
      \]
    \end{column}
  \end{columns}
\end{frame}

\begin{frame}
  \frametitle{Nomenclature}
  \tiny
  \begin{itemize}
    \nomenclature{$\varepsilon_{ma}$}{macro-porosity ($0.35$)}
    \nomenclature{$\varepsilon_{mi}$}{micro-porosity ($0.25$)}
    \nomenclature{$\varepsilon_{FC}$}{porosity of the flow channel spacer ($0.75$)}
    \nomenclature{$c$}{local salt concentration},
    \nomenclature{$c_{ions,\,mi}$}{ionic concentration in micro-pores}.
    \nomenclature{$\dot{S}$}{source/sink term}
    \nomenclature{$D_{eff}$}{effective salt diffusivity}\footnote{
    calculated based on porosity and bulk salt diffusivity, $D_0 = \expnumber{1.61}{-5}$ as $D_{eff} = D_0 \varepsilon^{1.5}$},
    \nomenclature{$D_{eff,FC}$}{effective salt diffusion coefficient in the flow
    \nomenclature{$F$}{Faraday Constant},
    \nomenclature{$\rho_e$}{electronic charge density},
    \nomenclature{$\rho_{mi}$}{ionic charge density},
    \nomenclature{$\rho_{chem}$}{surface immobile charge (4 $\mathrm{C/cm^3}$)},
    \nomenclature{$\Delta \phi_S$}{Stern Layer potential drop},
    \nomenclature{$\Delta \phi_D$}{Donnan Layer potential drop},
    \nomenclature{$\rho_{electrode}$}{electrode density (0.4664 $\mathrm{\sfrac{g}{cm^3}}$)},
    \nomenclature{$\kappa_{eff}$}{local effective
      ionic-conductivity}\footnote{It depends on the porosity and the
    corresponding bulk value: $\kappa_{eff} = \kappa_0\varepsilon^{1.5}$.
    According to Nernst-Einstein relation: $\kappa_0 = \sfrac{2 D_0 c F e}{k_B
      T}$},
    \nomenclature{$i_L$}{leakage current per unit electrode
      volume}\footnote{Due to parasitic faradaic reactions on the electrode}.
    \nomenclature{$\vec{v_s}$}{superficial velocity (assumed uniform)},
      channel}
  \end{itemize}
  \normalsize
\end{frame}

\begin{frame}
  \frametitle{Assumptions}
  \begin{itemize}
  \item Since the {\it\color{blue} direction of the current} is
    \textbf{perpendicular} to the {\it\color{blue} direction of the flow}, a
    \textbf{two dimensional} model is formulated for the continuous-flow
    operation scheme.
  \item In the model of \textbf{pulse-flow operation}, the effect of
    \textit{advection} can be {\color{red} neglected}, If we assume
    \begin{itemize}
    \item the event of pulsing happens \textbf{instantaneously}.
    \item This assumption {\it simplifies} the numerical implementation for
      pulse-flow operation and yeilds a {\color{red} one-dimensional}
      problem\footnote{Because the variation along the direction of the flow
      does not occur}.
    \end{itemize}
    \item Some other assumptins related to the faradaic reactions\dots
  \end{itemize}
\end{frame}

\begin{frame}
  \frametitle{Boundary and Initial Conditions}
  \begin{itemize}
  \item<1> {\bf Ionic flux} (excluding advective contributions) is set to zero at
    all surfaces except at flow-channel inlet.
  \item<2> {\bf Constant concentration} boundary condition at the flow-channel
    inlet.
  \item<3> In the {\it absence of flow}\footnote<.(1)->{For the one-dimensional model of
  pulse-flow operation} a {\bf null-flux} condition is imposed at the flow channel
    inlet. (?)
  \item<4> The {\bf total electronic current} at the {\it current collector
    surface} of the positive/negative electrode is constrained to the {\it total
    externally applied current}, while the {\bf total ionic current} is set to
    {\it zero} on these surfaces.
  \item<5> In all simulations we initialize the {\bf salt concentration} across
    the entire cell to the {\color{red} chosen in-fluent concentration}, and
    {\bf charge density} within the porous electrodes is initially set to
    {\color{red} zero}.
    \begin{itemize}
    \item From these initial conditions, initial cell voltage and ionic
      concentration inside micropores were calculated.
    \end{itemize}
  \end{itemize}
\end{frame}
